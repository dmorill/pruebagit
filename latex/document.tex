\documentclass[10pt,oneside]{article}   	
\usepackage{geometry}                		
\geometry{a4paper}                   		
\usepackage[latin1]{inputenc}
\usepackage{amsmath}
\usepackage{multirow}
\usepackage{amssymb}
\usepackage{csquotes}
\usepackage{lmodern}% http://ctan.org/pkg/lm
\usepackage{rotating}%
\usepackage{graphicx}											
\usepackage{amssymb}
\usepackage{authblk}



\usepackage[
	url = false,
	style = authoryear,
	hyperref = true,
	backref = true,
	backend = bibtex, % Otra opci?n es 'backend = bibtex'. como estaba biber
	doi = false,
	dashed=false,
]{biblatex}
\bibliography{references}
\usepackage[
	{colorlinks},
	{linkcolor=blue},	%colorEnlace
	{citecolor=blue},
	{urlcolor=blue},
	{bookmarksnumbered},
	{breaklinks},
]{hyperref}

\usepackage[spanish]{babel}

%

\title{Soluci�n parcial 1 optimizaci�n maestr�a}
\author[]{Daniel Morillo Torres}
\affil[]{Pontificia Universidad Javeriana Cali}
\renewcommand\Authands{, }
\date{}									

\begin{document}
	\maketitle
	
\section{1. Problema de la excavaci�n}
	Sea $D_i$ la duraci�n de la extracci�n de la piedra $i$ y $T_{max}=25$ una cota superior del problema y $SUCE$ es el conjunto de todas las parejas $(i,j)$ que representan que la actividad $j$ precede a la actividad $i$. 
	
	$SUCE=\{(6,1),(6,2),(7,2),(7,3),(8,3),(8,4),(9,4),(9,5),(10,6),(11,6),(11,7),(12,8),(12,9),$  $(13,9),(14,11),(15,14),(15,16),(16,12),(17,14),(18,14),(18,15),(19,15),(19,16),(20,16)\}$
	
	Las variables de decisi�n se definen a continuaci�n:
	\begin{equation}
		x_{it} = \left\lbrace
		\begin{array}{ll}
		1 & \textup{si la piedra $i$ se extrae en el tiempo $t$}\\
		0 & \textup{En otro caso}
		\end{array}
		\right.
	\end{equation}
	
	FO:
	\begin{equation}
		min z = \sum_{i=1}^{20}\sum_{t=1}^{T_{max}} \left[ x_{it}*(t+D_i)\right]
	\end{equation}
	s.t.
	\begin{equation}
		\sum_{t=1}^{T_{max}} x_{it} \leq 1 \;\;\; \forall i \in {1,..,20}
	\end{equation}

	\begin{equation}
		\sum_{i=17}^{20}\sum_{t=1}^{T_{max}} x_{it} = 1
	\end{equation}
	
	\begin{equation}
		\sum_{t=1}^{T_{max}} x_{jt}*(t+D_j) \leq M*(1-\sum_{t=1}^{T_{max}}x_{it}) + \sum_{t=1}^{T_{max}} x_{it}*t \;\;\; \forall (i,j) \in SUCE
	\end{equation}
	
	\begin{equation}
		\sum_{t=1}^{T_{max}} x_{jt} \geq \sum_{t=1}^{T_{max}} x_{it} \;\;\; \forall (i,j) \in SUCE
	\end{equation}
	
	Soluci�n:
	$z = 52$ escape completado en el tiempo $13$.\\
	$x_{1,1} = 1$	\\
	$x_{2,1} = 1$	\\
	$x_{3,1} = 1$	\\
	$x_{6,3} = 1$	\\
	$x_{7,4} = 1$	\\
	$x_{11,7} = 1$	\\
	$x_{14,8} = 1$	\\
	$x_{17,11} = 1$	\\
	
	Esta es una modificaci�n para que la vea caro linda.
	
	
\section{2. Problema de los 3 centros de distribuci�n}

Variables y soluci�n: Sea $C_{ijk}$ los costos asociados de transporte, $O_{i}$ la oferta de la planta $i$, $D_k$ la demanda de los mercados $k$. Las variables de decisi�n se definen a continuaci�n:
\begin{equation}
	x_{ijk} = \textup{env�o de art�culos desde la planta $i$ al centro de distribuci�n $j$ y finalmente al merado $k$.}
\end{equation}

\begin{equation}
	y_{j} = \left\lbrace
	\begin{array}{ll}
	1 & \textup{si el CD $j$ se construye}\\
	0 & \textup{En otro caso}
	\end{array}
	\right.
\end{equation}

FO:
\begin{equation}
	min z = \sum_{i=1}^{4}\sum_{j=1}^{3}\sum_{k=1}^{8} x_{ijk}*C_{ijk}
\end{equation}
s.t.
\begin{equation}
	\sum_{i=1}^{4}\sum_{j=1}^{3} x_{ijk} \geq D_k \;\;\; \forall k \in {1,..,8}
\end{equation}

\begin{equation}
	\sum_{j=1}^{3}\sum_{k=1}^{8} x_{ijk} \leq O_{i} \;\;\; \forall i \in {1,..,4} 
\end{equation}

\begin{equation}
	\sum_{i=1}^{4}\sum_{k=1}^{8} x_{ijk} \leq y_{j}*M \;\;\; \forall j \in {1,..,3}
\end{equation}

\begin{equation}
	\sum_{j=1}^{3} y_{j} = 2
\end{equation}

	Soluci�n:
$z = 621$\\
\begin{table}[h]
	\begin{center}
		\begin{tabular}{ll}
			\hline
			$x_{1,2,2} = 9$ & $y_{2} = 1$	\\
			$x_{1,2,7} = 11$ & 	$y_{3} = 1$	\\
			$x_{2,2,1} = 10$ &  \\
			$x_{2,2,4} = 8$	&  \\
			$x_{2,2,5} = 14$ & 	\\
			$x_{2,2,8} = 18$	& \\
			$x_{3,2,4} = 15$ & \\
			$x_{3,3,3} = 5$ & \\
			$x_{3,3,6} = 10$ & \\
			$x_{4,2,2} = 6$	& \\
			$x_{4,2,8} = 10$ & \\
			\hline
		\end{tabular}
		\label{tabla:sencilla}
	\end{center}
\end{table}

\begin{equation}
P(X=k)=\frac{\lambda^ke^-{\lambda}}{k!}, \quad \lambda \text{es la tasa de ocurrencia.}
\end{equation}

	% Bibliograf�a
	\bibitemsep = 3ex
	\bibhang = 2em
	
	\printbibliography[heading=bibintoc,title=\bibname]
	
\section{Secci�n nueva para la prueba}

\begin{equation}
	x = 20
\end{equation}



\end{document}   